%% Generated by Sphinx.
\def\sphinxdocclass{report}
\documentclass[letterpaper,10pt,english]{sphinxmanual}
\ifdefined\pdfpxdimen
   \let\sphinxpxdimen\pdfpxdimen\else\newdimen\sphinxpxdimen
\fi \sphinxpxdimen=.75bp\relax

\PassOptionsToPackage{warn}{textcomp}
\usepackage[utf8]{inputenc}
\ifdefined\DeclareUnicodeCharacter
% support both utf8 and utf8x syntaxes
\edef\sphinxdqmaybe{\ifdefined\DeclareUnicodeCharacterAsOptional\string"\fi}
  \DeclareUnicodeCharacter{\sphinxdqmaybe00A0}{\nobreakspace}
  \DeclareUnicodeCharacter{\sphinxdqmaybe2500}{\sphinxunichar{2500}}
  \DeclareUnicodeCharacter{\sphinxdqmaybe2502}{\sphinxunichar{2502}}
  \DeclareUnicodeCharacter{\sphinxdqmaybe2514}{\sphinxunichar{2514}}
  \DeclareUnicodeCharacter{\sphinxdqmaybe251C}{\sphinxunichar{251C}}
  \DeclareUnicodeCharacter{\sphinxdqmaybe2572}{\textbackslash}
\fi
\usepackage{cmap}
\usepackage[T1]{fontenc}
\usepackage{amsmath,amssymb,amstext}
\usepackage{babel}
\usepackage{times}
\usepackage[Bjarne]{fncychap}
\usepackage{sphinx}

\fvset{fontsize=\small}
\usepackage{geometry}

% Include hyperref last.
\usepackage{hyperref}
% Fix anchor placement for figures with captions.
\usepackage{hypcap}% it must be loaded after hyperref.
% Set up styles of URL: it should be placed after hyperref.
\urlstyle{same}

\addto\captionsenglish{\renewcommand{\figurename}{Fig.}}
\addto\captionsenglish{\renewcommand{\tablename}{Table}}
\addto\captionsenglish{\renewcommand{\literalblockname}{Listing}}

\addto\captionsenglish{\renewcommand{\literalblockcontinuedname}{continued from previous page}}
\addto\captionsenglish{\renewcommand{\literalblockcontinuesname}{continues on next page}}
\addto\captionsenglish{\renewcommand{\sphinxnonalphabeticalgroupname}{Non-alphabetical}}
\addto\captionsenglish{\renewcommand{\sphinxsymbolsname}{Symbols}}
\addto\captionsenglish{\renewcommand{\sphinxnumbersname}{Numbers}}

\addto\extrasenglish{\def\pageautorefname{page}}





\title{StepsRiverNetwork}
\date{Sep 17, 2019}
\release{0.9}
\author{}
\newcommand{\sphinxlogo}{\vbox{}}
\renewcommand{\releasename}{Release}
\makeindex
\begin{document}

\pagestyle{empty}
\maketitle
\pagestyle{plain}
\sphinxtableofcontents
\pagestyle{normal}
\phantomsection\label{\detokenize{index::doc}}
\begin{figure}[htbp]
\centering

\noindent\sphinxincludegraphics[width=7.5cm,height=3cm]{{logo}.png}
\end{figure}




\chapter{Background}
\label{\detokenize{index:background}}
StepsRiverNetwork simulates in-stream environmental fate processes of pesticides for
an entire river network of a catchment. Efate processes are calculated for each single
reach and transport across the entire catchment in an explicit timestep of minutes.

\begin{figure}[htbp]
\centering

\noindent\sphinxincludegraphics{{concept}.png}
\end{figure}

For simulating environmental processes the
STEPS1-2-3-4 appraoch by M. Klein (\sphinxurl{http://publica.fraunhofer.de/documents/N-73445.html}) has been implemented. The key features of STEPS-1-2-3-4 are as follow:
\begin{itemize}
\item {} 
A substance is defined by KOC and DT50.

\item {} 
The system consists of one water layer and two sediment layer.

\item {} 
Degradation is calculated by by DT50 in water and sediment.

\item {} 
Sorption / desorption are calculated by the gradient between water layer and pore water of sediment.

\end{itemize}

\begin{figure}[htbp]
\centering

\noindent\sphinxincludegraphics{{concept_steps1234}.png}
\end{figure}

STEPS1-2-3-4 solely simulates efate processes and hydrological data and compound inputs must be calculated in advance. Having all data at hand efat and solute flux across the whole river network is calculated for each timestep in relation to the followign scheme.

\begin{figure}[htbp]
\centering

\noindent\sphinxincludegraphics{{concept_steps1234_processes}.png}
\end{figure}


\chapter{Input data}
\label{\detokenize{index:input-data}}
All input data files must be located in the respective project folder, except the Runlist.


\section{RunList}
\label{\detokenize{index:runlist}}
\begin{figure}[htbp]
\centering

\noindent\sphinxincludegraphics{{data_input_RunList}.png}
\end{figure}


\section{ReachList}
\label{\detokenize{index:reachlist}}
\begin{figure}[htbp]
\centering

\noindent\sphinxincludegraphics{{data_input_ReachList}.png}
\end{figure}


\section{CatchmentList}
\label{\detokenize{index:catchmentlist}}
\begin{figure}[htbp]
\centering

\noindent\sphinxincludegraphics{{data_input_CatchmentList}.png}
\end{figure}


\section{SprayDriftList}
\label{\detokenize{index:spraydriftlist}}
\begin{figure}[htbp]
\centering

\noindent\sphinxincludegraphics{{data_input_SprayDriftList}.png}
\end{figure}


\section{HydroList}
\label{\detokenize{index:hydrolist}}
The HydroList can be provided by two alternative file formats.
\begin{enumerate}
\def\theenumi{\arabic{enumi}}
\def\labelenumi{\theenumi )}
\makeatletter\def\p@enumii{\p@enumi \theenumi )}\makeatother
\item {} 
CSV

\end{enumerate}

\begin{figure}[htbp]
\centering

\noindent\sphinxincludegraphics{{data_input_reach}.png}
\end{figure}
\begin{enumerate}
\def\theenumi{\arabic{enumi}}
\def\labelenumi{\theenumi )}
\makeatletter\def\p@enumii{\p@enumi \theenumi )}\makeatother
\setcounter{enumi}{1}
\item {} 
HDF5

\end{enumerate}

In this case, the file must called “HydroList.h5”. It must contain
three datasets for ‘area’ (wet surface, m2),
‘flow’ (m3/day) and ‘volume’ (m3) according to the parameter described above. Each dataset
must be  a two-dimensional array (shape: {[}ntime,nreaches{]}).


\chapter{Output data}
\label{\detokenize{index:output-data}}

\section{Environmental fate}
\label{\detokenize{index:environmental-fate}}\begin{enumerate}
\def\theenumi{\arabic{enumi}}
\def\labelenumi{\theenumi )}
\makeatletter\def\p@enumii{\p@enumi \theenumi )}\makeatother
\item {} 
CSV

\end{enumerate}

A summary CSV-file with a hourly timestep is optionally created which includes the
hydrological and efate data. The hourly summary can be calculated as minimum, maximum or
mean per hour.

\begin{figure}[htbp]
\centering

\noindent\sphinxincludegraphics{{data_output_reach}.png}
\end{figure}
\begin{enumerate}
\def\theenumi{\arabic{enumi}}
\def\labelenumi{\theenumi )}
\makeatletter\def\p@enumii{\p@enumi \theenumi )}\makeatother
\setcounter{enumi}{1}
\item {} 
HDF5

\end{enumerate}

Up to five state variables (MASS\_SW,MASS\_SED,MASS\_SED\_DEEP,PEC\_SW,PEC\_SED)
are saved in a seprated hdf5-file. Each file contains a dataset named by the respective
parameter (eg. steps\_MASS\_SW.h5 \textendash{}\textgreater{} dset=”MASS\_SW”) with a two-dimensional array
(shape: {[}ntime,nreaches{]}).


\chapter{Quick user guide}
\label{\detokenize{index:quick-user-guide}}
The model can be executed by using a Python script and calling the related functions:

\fvset{hllines={, ,}}%
\begin{sphinxVerbatim}[commandchars=\\\{\},numbers=left,firstnumber=1,stepnumber=1]
\PYG{k+kn}{import} \PYG{n+nn}{os}
\PYG{k+kn}{import} \PYG{n+nn}{argparse}
\PYG{k+kn}{from} \PYG{n+nn}{pathlib} \PYG{k+kn}{import} \PYG{n}{Path}
\PYG{k+kn}{from} \PYG{n+nn}{RunFactory} \PYG{k+kn}{import} \PYG{n}{RunFactory}

\PYG{k}{def} \PYG{n+nf}{get\PYGZus{}fdir}\PYG{p}{(}\PYG{n}{subfolder}\PYG{p}{)}\PYG{p}{:}
  \PYG{l+s+sd}{\PYGZdq{}\PYGZdq{}\PYGZdq{}}
\PYG{l+s+sd}{  Return current working directoy and add subfolder to path.}
\PYG{l+s+sd}{  \PYGZdq{}\PYGZdq{}\PYGZdq{}}
  \PYG{n}{fdir} \PYG{o}{=} \PYG{n}{os}\PYG{o}{.}\PYG{n}{path}\PYG{o}{.}\PYG{n}{abspath}\PYG{p}{(}\PYG{n}{os}\PYG{o}{.}\PYG{n}{path}\PYG{o}{.}\PYG{n}{join}\PYG{p}{(}
                  \PYG{n}{os}\PYG{o}{.}\PYG{n}{path}\PYG{o}{.}\PYG{n}{dirname}\PYG{p}{(}\PYG{n}{Path}\PYG{p}{(}\PYG{n+nv+vm}{\PYGZus{}\PYGZus{}file\PYGZus{}\PYGZus{}}\PYG{p}{)}\PYG{o}{.}\PYG{n}{parent}\PYG{p}{)}\PYG{p}{,}
                  \PYG{o}{*}\PYG{n}{subfolder}\PYG{p}{)}\PYG{p}{)}
  \PYG{k}{return} \PYG{n}{fdir}

\PYG{n}{FLAGS} \PYG{o}{=} \PYG{n+nb+bp}{None}

\PYG{k}{if} \PYG{n+nv+vm}{\PYGZus{}\PYGZus{}name\PYGZus{}\PYGZus{}} \PYG{o}{==} \PYG{l+s+s2}{\PYGZdq{}}\PYG{l+s+s2}{\PYGZus{}\PYGZus{}main\PYGZus{}\PYGZus{}}\PYG{l+s+s2}{\PYGZdq{}}\PYG{p}{:}

  \PYG{c+c1}{\PYGZsh{} get command line arguments or use default}
  \PYG{n}{parser} \PYG{o}{=} \PYG{n}{argparse}\PYG{o}{.}\PYG{n}{ArgumentParser}\PYG{p}{(}\PYG{p}{)}
  \PYG{n}{parser}\PYG{o}{.}\PYG{n}{add\PYGZus{}argument}\PYG{p}{(}\PYG{l+s+s1}{\PYGZsq{}}\PYG{l+s+s1}{\PYGZhy{}\PYGZhy{}folder}\PYG{l+s+s1}{\PYGZsq{}}\PYG{p}{,}\PYG{n+nb}{type}\PYG{o}{=}\PYG{n+nb}{str}\PYG{p}{,} \PYG{n}{default}\PYG{o}{=}\PYG{n}{get\PYGZus{}fdir}\PYG{p}{(}\PYG{p}{[}\PYG{l+s+s2}{\PYGZdq{}}\PYG{l+s+s2}{projects}\PYG{l+s+s2}{\PYGZdq{}}\PYG{p}{]}\PYG{p}{)}\PYG{p}{,}
                      \PYG{n}{help}\PYG{o}{=}\PYG{l+s+s1}{\PYGZsq{}}\PYG{l+s+s1}{Path of project folder.}\PYG{l+s+s1}{\PYGZsq{}}\PYG{p}{)}
  \PYG{n}{parser}\PYG{o}{.}\PYG{n}{add\PYGZus{}argument}\PYG{p}{(}\PYG{l+s+s1}{\PYGZsq{}}\PYG{l+s+s1}{\PYGZhy{}\PYGZhy{}runlist}\PYG{l+s+s1}{\PYGZsq{}}\PYG{p}{,}\PYG{n+nb}{type}\PYG{o}{=}\PYG{n+nb}{str}\PYG{p}{,}\PYG{n}{default}\PYG{o}{=}\PYG{l+s+s1}{\PYGZsq{}}\PYG{l+s+s1}{\PYGZsq{}}\PYG{p}{,}
                      \PYG{n}{help}\PYG{o}{=}\PYG{l+s+s1}{\PYGZsq{}}\PYG{l+s+s1}{Name of model run}\PYG{l+s+s1}{\PYGZsq{}}\PYG{p}{)}
  \PYG{n}{FLAGS}\PYG{p}{,} \PYG{n}{unparsed} \PYG{o}{=} \PYG{n}{parser}\PYG{o}{.}\PYG{n}{parse\PYGZus{}known\PYGZus{}args}\PYG{p}{(}\PYG{p}{)}

  \PYG{c+c1}{\PYGZsh{} create run factory}
  \PYG{n}{runfactory} \PYG{o}{=} \PYG{n}{RunFactory}\PYG{p}{(}\PYG{n}{FLAGS}\PYG{o}{.}\PYG{n}{folder}\PYG{p}{,}\PYG{n}{FLAGS}\PYG{o}{.}\PYG{n}{runlist}\PYG{p}{)}

  \PYG{c+c1}{\PYGZsh{} setup model runs}
  \PYG{n}{runfactory}\PYG{o}{.}\PYG{n}{setup}\PYG{p}{(}\PYG{p}{)}

  \PYG{c+c1}{\PYGZsh{} conduct simulations}
  \PYG{n}{runfactory}\PYG{o}{.}\PYG{n}{run}\PYG{p}{(}\PYG{n}{printres}\PYG{o}{=}\PYG{n+nb+bp}{False}\PYG{p}{)}
\end{sphinxVerbatim}

… or by using a batchfile by defining the variables ‘\textendash{}folder’ and ‘\textendash{}runlist’ which
area the the project folder and the name of the project:

\fvset{hllines={, ,}}%
\begin{sphinxVerbatim}[commandchars=\\\{\},numbers=left,firstnumber=1,stepnumber=1]
@echo off
\PYG{n+nb}{set} \PYG{n+nv}{script}\PYG{o}{=}\PYGZpc{}cd\PYGZpc{}/bin/main.py
\PYG{n+nb}{set} \PYG{n+nv}{python}\PYG{o}{=}\PYGZpc{}cd\PYGZpc{}/bin/Python/python.exe
call \PYGZpc{}python\PYGZpc{} \PYGZpc{}script\PYGZpc{} \PYGZhy{}\PYGZhy{}folder c:/projects/test/ \PYGZhy{}\PYGZhy{}runlist testrun
pause
\end{sphinxVerbatim}


\chapter{About}
\label{\detokenize{index:about}}
The tool is a development in a project by Bayer AG and knoell Germany GmbH.

Sebastian Multsch $^{\text{1}}$ , Stefan Reichenberger $^{\text{1}}$ ,Florian Krebs $^{\text{1}}$ , Thorsten Schad $^{\text{2}}$

$^{\text{1}}$ \sphinxhref{https://www.knoellconsult.com/enf}{knoell Germany GmbH}

$^{\text{2}}$ \sphinxhref{https://www.cropscience.bayer.de/}{Bayer AG, Research \& Development, Crop Science}



\renewcommand{\indexname}{Index}
\printindex
\end{document}